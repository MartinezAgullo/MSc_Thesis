During this master thesis a comprehensive review of the theoretical framework and of the current status of the analyses that have explored the  $Wtb$ vertex using $t$-channel single-top-quark events produced at the LHC has been done. Specifically, a new optimised set of selection cuts has been defined for the $t$-channel production at $\sqrt{s}$=13~TeV in the ATLAS detector of LHC. The variables considered in this analysis provide a fairly good background discrimination and, at the same time, do not employ kinematic properties of a single or two top quark decay products. Hence, their choice try to avoid potential kinematic biases in the angular distributions used in the measurement of the polarisation observables sensitive to the anomalous couplings.

The whole optimisation criteria described in Section~\ref{sec:ch05} leads to a proposal of the set of variables to be considered in the selection together with different values of the optimised cuts that correspond to different choices of the ratio $S/B$ together with the statistical significance (see Table~\ref{Table:FINAL}).


This work will benefit the Run~II ATLAS analyses of the $Wtb$ vertex using $t$-channel single-top-quark events, providing a baseline selection for the data sample to be used in these incoming analyses and ensuring an statistical optimisation. Further considerations will have to be taking into account when evaluating the systematic uncertainties at the end of the whole analysis chain.


%onclusions here:
%\begin{itemize}
%\item Exponer el resultado: set de cortes.
%\item Explicar qué grupos se benefician de este análisis. Explicar aquí o en la Sección 4 que uno de esos grupos requería el corte en $m_{lb}$. 
%\end{itemize}

%\vspace*{3 cm}

%Future work: (es importante contextualizar el trabajo futuro en el contexto del PhD)
%\begin{itemize}
%\item Build a neuronal network: Actualmente se dispone de una red neuronal para realizar este tipo de análisis pero está entrenada usando variables que están relacionadas quinematicamente con los productos de decaimiento del top y, por lo tanto, no la podemos usar (recordar que son variables en las que no se puede cortar porque las empleamos apra medir la polarización). Podría ser interesante entrenar una por nuestra cuenta (aunque este tipo de procedimientos se recomienda más para los experimentos en los que la señal es mucho más pequeña comparada con el bkg, ya que emplear un red neuronal es un aumento muy significativo de la sofisticación).
%\item Including the muon channel too.
%\end{itemize}

% Es importante conectarlo con las perspectivas del PhD


\newpage