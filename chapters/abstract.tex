\newpage
\pagestyle{fancy}
\lhead{\fontfamily{cmss}\selectfont \textbf{ Master’s Final Project }}
\rhead{\fontfamily{cmss}\selectfont \textbf{Master's Degree in Advanced Physics}}
\lfoot{\fontfamily{cmss}\selectfont \textbf{2016-2017}}
\rfoot{\thepage}
\cfoot{\textsc{ }}
\renewcommand{\headrulewidth}{1pt}
\renewcommand{\footrulewidth}{1pt}

%\begin{center}
%{\textsc{\Large Trabajo de Fin de Máster}}
%\end{center}
\renewcommand*\abstractname{Abstract}

\begin{abstract}

%%%%%%%%%%%%%%%%%%% VERSION 1 %%%%%%%%%%
%In this work we study the $Wtb$ vertex structure in $t$-channel single-top-quark production and decay. This vertex is very sensitive to the $A_{FB}^{N}$ asymmetry and may help to find new sources of $\mathcal{CP}$ violation. For this purpose, top-quark and $W$-boson polarization observables are measured. The analysed dataset of simulations corresponds corresponds	to	$13$~TeV centre-of-mass	energy proton-proton collisions with an integrated luminosity of	$35$~pb$^{-1}$,	recorded with the LHC's ATLAS detector during Run II.


%The event selection criteria must be optimized to enhance the quality of the data. In order to increase the signal significance and the signal to background ratio, we need stringent selection criteria. This is accomplished by applying different sets of cuts over the MC simulations and studying how the statistical parameters ($Signal$ $significance$ and $S/B$ ratio) evolve with the selected samples. Finally, a new selection is proposed using the variables: 
%top-quark mass, transverse energy of all final objects, the pseudorapidity of the light-jet, the pseudorapidity distance between the light-jet and the $b$-jet,  the reconstructed mass of the light-jet plus the top-quark and the invariant mass of the charged lepton and the $b$-tagged jet.
%$m_{top}$, $H_T$, $\eta_j$, $\Delta\eta_{j,top}$, $m_{j,top}$ and $m_{l,b}$.

%This process consist of identifying the most sensitive variables and in verifying through the study of correlations that adding conditions on them does not remove signal events from the variables that will be used for measurements.



%%%%%%%%%%%%%%%%%%% VERSION 2 %%%%%%%%%%
The top-quark is the heaviest known fundamental particle and probing its couplings with the other fundamental particles may open a window to physics beyond the Standard Model. Single top-quark production provides a unique way to study the coupling between the top-quark, the $W$-boson and the $b$-quark, since it involves the $Wtb$ vertex in both production and decay.

The $Wtb$ vertex structure has been explored through the measurement of top-quark and $W$-boson polarisation observables using $t$-channel single top-quark events produced in proton-proton collisions during LHC Run I. The most recent limits on anomalous couplings in the $Wtb$ vertex from the ATLAS and CMS experiments are reviewed.

Comprehensive studies of the $t$-channel selection requirements have been performed using Monte Carlo samples of signal and background events produced at $\sqrt{s}=13$~TeV and normalised to an integrated luminosity of 36.1 pb$^{-1}$. Starting from events containing one isolated lepton, large missing transverse momentum and exactly two jets, one being $b$-tagged, stringent selection requirements are proposed to discriminate $t$-channel single-top events from the background contributions. These selection requirements have been designed on the basis of an statistical optimisation and regardless of kinematic properties of the top decay products that could potentially bias early measurements of top-quark polarisation in the Run II.







% statistical parameters = significance and $S/B$
% variables that will be used for measurements =  los cosenos de las polariazaciones y lo de la helicidad
% adding conditions = imposing boundary conditions and priors

%In order to increase the signal to background ratio, we need to apply stringent selection criteria to discriminate the signal, composed by $t$-channel single tops events, from the background. It is necessary to keep in mind that this cannot be done at any price and that the conditions that give great $S/B$ ratios could exclude too many events; which would imply a loss of statistical significance.

%In this work is analysed the $Wtb$ vertex structure in $t$-channel single-top-quark production and decay. The analysed dataset correspond to $pp$ collisions at $\sqrt{s}=13$ TeV at the ATLAS detector of the LHC during Run II, with an integrated luminosity of $35$ $fb^{-1}$.

%* Introducir cosas de CP 
%* Develop optimization tool \\
%* Search of discriminant variables to improve the selection with respect to the one used with  8 TeV. \\
%* Deeper study of the most sensitive variables \\
%* Optimization of  the cuts in those variables. Selection criteria optimized in order to maximize the signal significance and reduce to a minimum the possible systematic uncertainties. \\



\end{abstract}




%\section*{Dedication}
%To mum

%\section*{Declaration}
%This dissertation is submitted in partial fulfilment of the requirements for the University of Valencia's M.Sc Degree of Advanced Physics. I declare that this thesis and the work presented in it are my own and has been generated by me as the result of my own original research. The contribution of any supervisors and others to the research and to the dissertation was consistent with normal supervisory practice. External contributions to the research  are acknowledged and referenced.

\section*{Acknowledgements}
%During the realization of this thesis I have benefit from the help of many others and hence I find necessary to acknowledge those who have supported me.
First and foremost I have to express my most sincere gratitude to my thesis advisor, Susana Cabrera, for her constant guidance, indefatigable coaching and for welcoming close collaboration throughout the project. Her supervision has been an enriching experience from which I have learnt a lot.

I would like to thank the ATLAS Silicon IFIC group. To Carmen García, José Enrique García and María José Costa, for being inclusive and supportive since the first day and for their advices. Also to Carlos Escobar for all his assistance and directions. I would also like to thank the PhD students Florencia Castillo, Laura Barranco and Óscar Estrada, and the freshmen Galo Gonzalvo, Jose Antonio Fernández and Marcos Martínez for their counsels.

I shall not forget the enjoyable collaboration and useful brainstorming of the $t$-channel polarisation analysis subgroup of ATLAS: A.Lleres (CNRS/IN2P3 Grenoble), M.~de~Beurs and M.Vreeswijk (NIKHEF National Institute for Subatomic Physics) and J.Mueller
and J.Boudreau (University of Pittsburgh) among others.

This work has been supported by a grant founded by the IFIC Severo Ochoa formation programme 2016-2017 for master students.
%Programa de Formación Severo Ochoa del IFIC - Curso 2017-18

% PhD students Laura Barranco and Óscar Estrada. 

%I want to thank to  Susana for... Also carlos Escobar... %Carmen García, MJ Costa and JEnrique for... t-chann polarization group <--- ESTO EH EIMPORATNTE. Laura and Oscar... Galo because teamwork..  I would also like to thank IFIC for granting me with Severo Ochoa... % Florencia que me ha salvado con el latex mil y una veces. %
%
Finally, I must express my very profound gratitude to my mother for providing me with unfailing support and continuous encouragement throughout my years of study. This accomplishment would not have been possible without her. Gràcies.

